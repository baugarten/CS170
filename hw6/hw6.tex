% Search for all the places that say "PUT SOMETHING HERE".


\documentclass[11pt]{article}
\usepackage{amsmath,textcomp,amssymb,geometry,graphicx,tkz-graph, qtree}

\def\Name{Ben Augarten}  % Your name
\def\Sec{106}  % Your discussion section
\def\Login{cs170-bo} % Your login

\def\pm{\begin{pmatrix}}
\def\epm{\end{pmatrix}}

\title{CS170--Fall 2012 --- Solutions to Homework 6}
\author{\Name, section \Sec, \texttt{\Login}}
\markboth{CS170--Spring 2012 Homework 6 \Name, section \Sec}{CS170--Spring 2012 Homework 6 \Name, section \Sec, \texttt{\Login}}
\pagestyle{myheadings}

\begin{document}
\maketitle
\begin{enumerate}
\item
\begin{enumerate}
\item[4.8] No this is not a valid method. Lets say there are two paths from $s$ to $t$. Even in the case of no negative edges, lets say these two paths have length $n$ and $m$, $n<m$. The first path of length $n$ goes through $10$ edges, the second through $1$. If we add a constant factor $k$ to each edge, then the first path now has cost $n+10k$, while the second has $m+k$. Well, now $n+10k > m+k$ so Professor F. Lake would get the incorrect answer using his algorithm. Of course, we can construct many counter examples that follow this pattern
\item[4.9]Yes Dijkstra's will work (assuming no edges enter $s$ as Prof Rao said on piazza). The lack of incoming edges to $s$ prevents a negative cycle. Now, any shortest path between $s$ and any $u$ has to go through an outgoing edge of $s$ exactly once (because $s$ has no incoming edges). Because of this, we can add a constant factor to every outgoing edge of $s$ as proposed in 4.8. This constant factor is added exactly once to each shortest path and makes all of the outgoing edges positive, so we know that Dijkstra's works on this graph. Because its added only once to each path, we don't run into problems as posed in 4.8. Finally, subtract the constant factor from the length of the final path found. 
\end{enumerate}
\newpage
\item Go through the shortest path tree and label each node on the graph with their shortest path. (This is trivial). Once we label all of the vertices with their potentially shortest paths, for all edges, $e$, between $(u,v)$, dist$(u) + l(e)>=$dist$(v)$ -- that is, a node can't be farther than all of its ancestors plus the length of the edge from the ancestor. Therefore, we can run one iteration of bellman-ford and if any vertex is updated, then we know that its not a shortest path tree. If not, then we know it is a shortest path tree by the same argument as bellman-ford being correct (we know its correct when there is no update preformed). This runs in linear time.
\newpage
\item
\begin{enumerate}
\item Cost: 19
\item 2 different MSTs
\item Added in this sequence: AE, EF, FB, FG, GH, CG, GD\\
For AE: Cut: A and B..H\\
For EF: Cut: AE and BCDFGH\\
For FB: Cut: AEF and BCDGH\\
For FG: Cut: AEFB and CDGH\\
For GH: Cut: AEFBG and CDH\\
For CG: Cut: AEFBGH and CD\\
For GD: Cut: AEFBGHC and D\\
\end{enumerate}
\newpage
\item
\begin{enumerate}
\item Why would you make me do this\\
I used code from rosettacode.org: http://rosettacode.org/wiki/Huffman\_coding\#Python\\
empty	111\\
e	010\\
a	1010\\
h	0001\\
i	0111\\
n	0110\\
o	1000\\
r	0000\\
s	0011\\
t	1100\\
c	00101\\
d	10111\\
l	10110\\
u	00100\\
b	100100\\
f	110100\\
g	100111\\
m	110111\\
p	100101\\
w	110101\\
y	100110\\
v	1101100\\
k	11011010\\
x	110110110\\
j	1101101110\\
q	11011011110\\
z	11011011111
\item 4.201
\item Entropy is 4.15, only slightly smaller than out solution, to be expected.
\item I don't think this is a the limit -- we can also assign sequences to especially common words or sequences of characters. I.e. the, an, or even (space)I(space) would be especially common sequences of characters that could get special encodings.
\end{enumerate}
\newpage
\item
\begin{enumerate}
\item Degree sequence 3,1,1,3
\item
\begin{enumerate}
\item If the neighbors of $v_1$ are not $v_2,..., v_{d_1+1}$, then $v$ does not share an edge with one of its first $d_1$ neighbors. But, in order for $v_1$ to have degree $d_1$ it must share edges with $d_1$ other vertices (no multiple edges). Therefore, there must be at least one vertex not in $v_2,..., v_{d_1+1}$ such that $\{v_1, v_j\}\in E$. Now we have just shown that there must be $\{v_1,v_i\} \notin E$. Now we must show that there has to exist a $u$ that doesn't share an edge with $v_j$. This is easy -- Because $v_1$ doesn't share edges with all $n$ neighbors, each vertex can have degree only as high as $v_1$. Therefore, its obvious that $v_j$ cannot have edges to all other vertices, there must exist a $u$ such that $v_j$ doesn't share an edge with $u$. There must exist a $u$ such that $v_i$ shares an edge with it and $v_j$ does not because $v_i$ has equal or higher degree than $v_j$. In either case, $v_j$ shares an edge with $v_1$, so it has $d_j-1$ other outgoing edges. $v_i$ has at least $d_j$ outgoing edges, one of which cannot go to the same vertices has one in the set of $d_j-1$ edges (the sets aren't equal). Therefore, there must exist a $u$ that shares an edge with $v_i$ and not with $v_j$.
\item First, connect $v_1$ with $v_i$ and remove the connection between $v_1$ and $v_j$. Now $v_j$ has degree $d_j-1$ and $v_i$ has degree $v_i+1$. The good news is we know that a certain $u$ exists that shares an edge with $v_i$ and not with $v_j$. We can simply change the edge $\{v_i,u\}$ to be $\{v_j,u\}$. Now the degrees are fixed and $(v_1,v_i)\in E$
\item We can repeat the above procedure, picking out $v_i$'s one at a time and replacing them with vertices that are connected to $v_1$ until all $v2,..., v_{d_1+1}$ are neighbors of $v_1$. This will produce a graph with the same degree sequence (because each iteration doesn't change the degree sequence) but $v_1$ neighbors come after it in the ordering
\end{enumerate}
\item Sort the degrees in non-decreasing order, much like we had in part b. Now we have $d_1,d_2,..., d_n$. For the elements $d_2,..., d_{d_1+1}$, decrement their degree by $1$. We can do this because these elements must have or we can rearrange the graph for them to share edges with $d_1$. Now remove $d_1$ and recurse on the remaining graph.\\
If we get to an element that has degree higher than the remaining number of elements, then the graph doesn't exist. If we get to the last element and its degree is $0$ then awesome! the graph exists, if not, the graph doesn't exist.\\
Correctness: If we got a nonzero degree value for the last element, then there are either too few incoming edges (positive) or too many incoming edges (negative). Therefore, the degree sequence couldn't possibly exist because the higher degrees either connect too many times with the lower degrees or not enough. If we get a $0$ for the last degree, then the number of incoming edges matches the number of outgoing edges for each node, so the handshaking lemma holds for each vertex, so the graph must exist. If we reach a node that has degree higher than the remaining number of nodes, that means it has too many outgoing edges left and couldn't possibly connect to all of the remaining nodes, so the graph cannot exist.\\
Run time: each time we have to sort $n$ elements. We run the algorithm $n$ times, each time looking at at most $n$ elements, so it runs in $O(n^3\log(n))$ time. The sort should be relatively fast though, considering we shouldn't change the ordering much between iterations. 
\end{enumerate}
\end{enumerate}
\end{document}
